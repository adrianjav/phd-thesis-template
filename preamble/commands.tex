% ===================================================================
% MATH
% ===================================================================
\newcommand{\verylongrightarrow}{\xrightarrow{\hspace*{1.5cm}}}
\newcommand{\explainmath}[1]{\ensuremath{&&{\footnotesize\text{#1}}}} % Only works in align environments

% punctuation of equations, see for example the first answer in
% https://www.reddit.com/r/LaTeX/comments/5xnzg7/correct_grammar_for_putting_an_equation/
\newcommand{\equationPunctuation}[1]{\,{#1}}

% ===================================================================
% REFERENCES
% ===================================================================

% https://davidyat.es/2016/07/27/writing-a-latex-macro-that-takes-a-variable-number-of-arguments/
% \newcommand{\subfigref}[1]{\textbf{(\subref{#1}\subfigrefchecknextarg)}}
% \newcommand{\subfigrefchecknextarg}{\@ifnextchar\bgroup{\subfigrefgobblenextarg}{}}
% \newcommand{\subfigrefgobblenextarg}[1]{, \subref{#1}\@ifnextchar\bgroup{\subfigrefgobblenextarg}{}}

\newcommand{\subfigref}[1]{\textbf{(\subref{#1})}}

% ===================================================================
% COMMON ABBREVIATONS


\makeatletter
\newcommand{\PhD}{Ph.D.\@\xspace}
\makeatother
\newcommand*{\citeeg}{\textit{e.\nobreak\hairsp{}g.},}		% Special \eg command used in citations (doesn't result in weird space)
\newcommand*{\cf}{cf.\@\xspace}
\newcommand{\versus}{\textit{v.\nobreak\hairsp{}s.}\xspace}
\newcommand{\randomvariable}{\textit{r.\nobreak\hairsp{}v.}\xspace} % rv is already used
\newcommand{\randomvariables}{\textit{r.\nobreak\hairsp{}v.\nobreak\hairsp{}s}\xspace} % rv is already used
\newcommand{\rhs}{right-hand side\xspace}

\usepackage{nth}						% For things like 2nd, 3rd, etc.
\usepackage{xstring}

\let\oldnth\nth
\renewcommand{\nth}[1]{%
    \IfInteger{#1}{\oldnth{#1}}{\hbox{#1-th}}%
%    \hbox{$#1$\ifnum1=0#1\relax st\else\ifnum2=0#1\relax nd\else\ifnum3=0#1\relax rd\else th\fi\fi\fi}
}  % it does the job

% ===================================================================
% COLORED ITEMS

% Coloured dots
\DeclareRobustCommand{\colordot}[1]{%
	\begin{tikzpicture}[baseline=(a.south)]
			\node[circle, scale=0.75,color=white, fill=#1] (a) {};
	\end{tikzpicture}%
}

% Coloured square
\DeclareRobustCommand{\colorsquare}[1]{%
	\begin{tikzpicture}[baseline=(a.south)]
			\node[rectangle, scale=0.9,color=white, fill=#1] (a) {};
	\end{tikzpicture}%
}

% Coloured line
\DeclareRobustCommand{\colorline}[1]{%
	\begin{tikzpicture}
		\raisebox{1.5pt}{
			\draw[#1,solid,line width=1.5pt] (0,0) -- (1em,0);
		}
	\end{tikzpicture}%
}

% Coloured arrow
\DeclareRobustCommand{\colorarrow}[1]{%
	\begin{tikzpicture}
		\raisebox{2.5pt}{
			\draw[#1, thick, -stealth] (0,0) -- (1em,0);
		}
	\end{tikzpicture}%
}

% Coloured shade
\newcommand{\colorshade}[1]{\textcolor{#1}{\ding{122}}}


%%% Local Variables:
%%% mode: latex
%%% TeX-master: "../thesis"
%%% End:
