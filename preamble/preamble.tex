% ===================================================================
% GENERAL PREAMBLE

\documentclass[
				a4paper,
				fontsize=11pt,
				twoside=true,
				numbers=noenddot,
        open=any,
        secnumdepth=2,
%				draft % Shows black boxes next to overful hboxes (ignore the errors)
			   ]{preamble/thesis}

\usepackage[ngerman,english]{babel} 		% English language/hyphenation
\usepackage[dissertation, internal]{preamble/tuetitle}


% ===================================================================
% COLORS

\usepackage{xcolor} 				% To define colors
% ===================================================================
% SAARLAND UNIVERSITY

\definecolor{SBred}{RGB}{200,34,84}
\definecolor{SBdark}{RGB}{1,40,63}
%\definecolor{SBgold}{RGB}{215,223,35}
\definecolor{SBgold}{RGB}{171,179,25}  % darker than the official
\definecolor{SBblue}{RGB}{0,72,119}

\colorlet{darkgreen}{green!50!black}

%%% Local Variables:
%%% mode: latex
%%% TeX-master: "../thesis"
%%% End:
			% Pre-define a bunch of often-used colors

\colorlet{maincolor}{TUblue}		% Main color, used for headings, etc.
\colorlet{secondcolor}{TUred}		% Secondary color, used for references, etc.
\colorlet{thirdcolor}{TUgold}		% Tertiary color, used for citations, etc.
\colorlet{fourthcolor}{TUgold}		% Fourth color, used for URLs, etc.
\colorlet{lightgraycolor}{TUdark!50}% A gray color, used for lines, etc.
\colorlet{darkgraycolor}{TUdark!80}	% A dark color, used for numbers, etc.

% Optimizer colors
\definecolor{adabelief}{HTML}{63FFA4}
\definecolor{amsbound}{HTML}{FFFF00}
\definecolor{amsgrad}{HTML}{A63D08}
\definecolor{adabound}{HTML}{FF8C00}
\definecolor{adadelta}{HTML}{735859}
\definecolor{adagrad}{HTML}{F3CC9A}
\definecolor{adam}{HTML}{F90004}
\definecolor{lookaheadmomentum}{HTML}{008B8B}
\definecolor{lookaheadradam}{HTML}{35CC38}
\definecolor{momentum}{HTML}{00CBFF}
\definecolor{nag}{HTML}{991bf5}
\definecolor{nadam}{HTML}{8E8E05}
\definecolor{radam}{HTML}{FF00FF}
\definecolor{rmsprop}{HTML}{000000}
\definecolor{sgd}{HTML}{1E3CFF}
\definecolor{otheropt}{HTML}{B2B2B2}

% DeepOBS colors
\colorlet{deepobsclass}{TUgold}
\colorlet{deepobscli}{TUdark}
\colorlet{deepobsdata}{TUblue}
\colorlet{deepobsscript}{TUred}

\colorlet{deepobsimageclass}{TUgold}
\colorlet{deepobsgenerative}{TUblue}
\colorlet{deepobsnlp}{TUred}
\colorlet{deepobstoy}{TUdark}

\colorlet{deepobssmall}{SNSgreen!45}
\colorlet{deepobslarge}{SNSorange!45}

\colorlet{deepobsgood}{SNSgreen!70}
\colorlet{deepobsbad}{TUred!70}

% Benchmark colors
\colorlet{lrschedules}{maincolor}
\colorlet{lrbg}{lightgraycolor!25}

% ===================================================================
% GRAPHICS

\usepackage{pifont}					% For special symbols like checkmarks
\usepackage[labelsep=colon]{caption}				% Options for caption
\DeclareCaptionFont{labelcolor}{\color{maincolor}}
\captionsetup{labelfont={labelcolor,bf}} 	% e.g. coloring the word "Figure" in caption
\usepackage{subcaption}				% For subcaptioning images
\usepackage{fancybox}				% Allows shadowbox around figure (for screenshot)

% path where the \includegrahics looks for figures
\graphicspath{
  {../repos/cockpit-paper/fig/03_scalar_deep/output/}
  {../repos/vivit-paper/tex/paper/}
}

\newcommand{\goldenRatio}{1.61803398875}
\newcommand{\goldenRatioInv}{0.61803398875}

% Tikz
\usetikzlibrary{
  shapes,
  pgfplots.groupplots,
  shadings,
  calc,
  arrows,
  backgrounds,
  colorbrewer,
  shadows.blur,
  external,
  shapes,
  shapes.arrows,
  shapes.symbols,
  arrows.meta,
  fit % For extending the bounding box
}

% PGFPlot
\usepackage{pgfplots}
\usepgfplotslibrary{groupplots,fillbetween}
\pgfplotsset{compat=1.15}

% Externalize (disabled by default)
\usepgfplotslibrary{external}
\tikzexternalize[prefix=tikz/, mode=list and make]
\tikzexternaldisable

% Create new lenght for pgfplots
\newlength\figureheight
\newlength\figurewidth
\setlength\figureheight{\textheight}
\setlength\figurewidth{\textwidth}

% width in figure* environments
\newlength{\thesiswidewidth}
\setlength{\thesiswidewidth}{457.8024pt}

\pdfsuppresswarningpagegroup=1		% Surpress the "PDF inclusion: multiple pdfs with apge group..." warning (https://tex.stackexchange.com/questions/76273/multiple-pdfs-with-page-group-included-in-a-single-page-warning)

% Customized shadow color for shadowbox
\makeatletter
\newcommand\Cshadowbox{\VerbBox\@Cshadowbox}
\def\@Cshadowbox#1{%
	\setbox\@fancybox\hbox{\fbox{#1}}%
	\leavevmode\vbox{%
		\offinterlineskip
		\dimen@=\shadowsize
		\advance\dimen@ .5\fboxrule
		\hbox{\copy\@fancybox\kern.5\fboxrule\lower\shadowsize\hbox{%
				\color{lightgraycolor}\vrule \@height\ht\@fancybox \@depth\dp\@fancybox \@width\dimen@}}%
		\vskip\dimexpr-\dimen@+0.5\fboxrule\relax
		\moveright\shadowsize\vbox{%
			\color{lightgraycolor}\hrule \@width\wd\@fancybox \@height\dimen@}}}
\makeatother

% ===================================================================
% FONTS

\usepackage[defaultsans]{lato}		% Sans Font: Lato (would allow for thin)
%\usepackage{gillius}				% Sans Font: Tufte font

\usepackage{anyfontsize}			% Allows arbitrary font sizes

% Headings should use sans serif font
\addtokomafont{part}{\normalfont\sffamily\color{maincolor}}

% Colored chapter number but not the chapter title
\addtokomafont{chapter}{\normalfont\sffamily}
\addtokomafont{chapterprefix}{\normalfont\sffamily\color{maincolor}}
\addtokomafont{section}{\normalfont\sffamily\color{maincolor}}
\addtokomafont{subsection}{\normalfont\sffamily\color{maincolor}}
\addtokomafont{subsubsection}{\normalfont\sffamily\color{maincolor}}

\addtokomafont{pagehead}{\normalfont\sffamily}	% Also use sans serif font in header

% Checkmarks
\newcommand{\cmark}{\ding{51}}%
\newcommand{\xmark}{\ding{55}}%

% Use Computer Modern Symbols for mathcal
\DeclareMathAlphabet{\mathcal}{OMS}{cmsy}{m}{n}
\SetMathAlphabet{\mathcal}{bold}{OMS}{cmsy}{b}{n}

% Fixes overful hboxes in references
% see https://tex.stackexchange.com/questions/171999/overfull-hbox-in-biblatex
\emergencystretch=1em


% ===================================================================
% REFERENCES
% ===================================================================

\usepackage{varioref}
\usepackage{hyperref}
\usepackage[capitalize,nameinlink,noabbrev]{cleveref}

% Customization of hyperref options
\hypersetup{
	unicode, % Use unicode for links
	pdfborder={0 0 0}, % Suppress border around pdf
	bookmarksdepth=section,
	bookmarksopen=true, % Expand the bookmarks as soon as the pdf file is opened
	%bookmarksopenlevel=4,
	%	linktoc=all, % Toc entries and numbers links to pages
	linktocpage=true,  % Only the page number links to pages
	breaklinks=true,
	colorlinks=true,
	citecolor = thirdcolor,
	linkcolor = secondcolor,
	urlcolor = fourthcolor,
}


% ===================================================================
% MATH
% ===================================================================

% Fix "Too many math alphabets" error (https://tex.stackexchange.com/a/243541)
\newcommand\hmmax{0}
\newcommand\bmmax{0}

\usepackage{amsfonts}				% blackboard math symbols
\usepackage{amsmath}				% for all basic math operations
\usepackage{amssymb}
\usepackage{nicefrac}				% compact symbols for 1/2, etc.
%\usepackage{ntheorem}				% required by kaobook (otherwise \theoremstyle is not defined)
\usepackage{siunitx}  				% SI units (used for run times)
\usepackage{xfrac}					% For slanted fractions "sfrac" like the nicefrac
\usepackage{bm}						% Offers \bm to create bold math (used by Goodfellow)
\usepackage{derivative}				% Handy commands for typesetting derivatives
\usepackage{mathtools}				% for vcentcolon a centered colon used in \eqdef
% Load mathematical packages for theorems and related environments
\usepackage[framed=true,definitionbackground=lightgraycolor!25,theorembackground=lightgraycolor!25,
examplebackground=lightgraycolor!25]{preamble/kaotheorems}

% Bugfix: https://tex.stackexchange.com/questions/262142/thmtools-notebraces-bug
\makeatletter
%%% from amsthm.sty
\def\thmhead@plain#1#2#3{%
	\thmname{#1}\thmnumber{\@ifnotempty{#1}{ }\@upn{#2}}%
	%%% the line below had (##3)
	\thmnote{ {\the\thm@notefont\thm@lparen#3\thm@rparen}}}

%%% from thm-amsthm.sty
\def\thmt@setheadstyle#1{%
	\thmt@style@headstyle{%
		\def\NAME{\the\thm@headfont ##1}%
		\def\NUMBER{\bgroup\@upn{##2}\egroup}%
		%%% the line below had (##3)
		\def\NOTE{\if=##3=\else\bgroup\thmt@space\the\thm@notefont\thm@lparen##3\thm@rparen\egroup\fi}%
	}%
	\def\thmt@tmp{#1}%
	\@onelevel@sanitize\thmt@tmp
	%\tracingall
	\ifcsname thmt@headstyle@\thmt@tmp\endcsname
	\thmt@style@headstyle\@xa{%
		\the\thmt@style@headstyle
		\csname thmt@headstyle@#1\endcsname
	}%
	\else
	\thmt@style@headstyle\@xa{%
		\the\thmt@style@headstyle
		#1%
	}%
	\fi
	%\showthe\thmt@style@headstyle
}
%%% the line below had (#3)
\def\thmt@embrace#1#2\thm@lparen#3\thm@rparen{#1#3#2}
%%% added for default
\def\thm@lparen{(}\def\thm@rparen{)}
\makeatother

\xspaceaddexceptions{]}

% \declaretheoremstyle[
% %spaceabove=.5\thm@preskip,
% %spacebelow=.5\thm@postskip,
% %headfont=\normalfont\bfseries,%\scshape,
% notefont=\bfseries,
% notebraces={ [}{]},
% bodyfont=\normalfont,
% %headformat={\NAME\space\NUMBER\space\NOTE},
% headpunct={},
% postheadspace=\newline,
% %prefoothook={\hfill\qedsymbol}
% ]{mytheoremstyle}

% \theoremstyle{mytheoremstyle}
% \declaretheorem[
% name=Definition,
% %refname={definition,definitions},
% refname={Definition,Definitions},
% Refname={Definition,Definitions},
% numberwithin=section,
% mdframed={
% 	style=mdfkao,
% 	backgroundcolor=lightgraycolor!25,
% 	%frametitlebackgroundcolor=\@theorembackground,
% },
% ]{thesisdefinition}


% \declaretheorem[name=Update Rule,
% refname={update rule,update rules},
% Refname={Update Rule,Update Rules},
% numberwithin=section,
% mdframed={
% 	style=mdfkao,
% 	backgroundcolor=lightgraycolor!25,
% %	%frametitlebackgroundcolor=\@theorembackground,
% },
% sibling=thesisdefinition]{thesisupdaterule}

% \declaretheorem[
% name=Theorem,
% refname={Theorem,Theorems},
% Refname={Theorem,Theorems},
% numberwithin=section,
% mdframed={
% 	style=mdfkao,
% 	backgroundcolor=lightgraycolor!25,
% 	%	%frametitlebackgroundcolor=\@theorembackground,
% },
% sibling=thesisdefinition
% ]{thesistheorem}

% ===================================================================
% CODE
% !!! Important to keep this after the definition of fancybox package
% 	  as it otherwise will create a weird bug !!!

% Inline code that looks similar to markdown inline code snippets
\newcommand{\inlinecode}[1]{%
	\begin{tikzpicture}[baseline=0ex]%
		\node[anchor=base,%
		text height=1em,%
		text depth=1ex,%
		inner ysep=0pt,%
		draw=darkgraycolor!30!white,%
		fill=lightgraycolor!20!white,%
		rounded corners=2pt] at (0,0) {\footnotesize\texttt{#1}};%
	\end{tikzpicture}%
}
% TikZ code contains fragile commands which throws errors when used in captions
% and section titles. This robust command can be used as drop-in
% (https://tex.stackexchange.com/a/56081)
\DeclareRobustCommand\robustInlinecode[1]{\inlinecode{#1}}

\usepackage{listings}				% For full code blocks

% Loads the lstlinebgrd package, but with some changes since it is broken:
% https://tex.stackexchange.com/questions/451532/recent-issues-with-lstlinebgrd-package-with-listings-after-the-latters-updates
\makeatletter
\let\old@lstKV@SwitchCases\lstKV@SwitchCases
\def\lstKV@SwitchCases#1#2#3{}
\makeatother
\usepackage{lstlinebgrd}
\makeatletter
\let\lstKV@SwitchCases\old@lstKV@SwitchCases

\lst@Key{numbers}{none}{%
	\def\lst@PlaceNumber{\lst@linebgrd}%
	\lstKV@SwitchCases{#1}%
	{none:\\%
		left:\def\lst@PlaceNumber{\llap{\normalfont
				\lst@numberstyle{\thelstnumber}\kern\lst@numbersep}\lst@linebgrd}\\%
		right:\def\lst@PlaceNumber{\rlap{\normalfont
				\kern\linewidth \kern\lst@numbersep
				\lst@numberstyle{\thelstnumber}}\lst@linebgrd}%
	}{\PackageError{Listings}{Numbers #1 unknown}\@ehc}}
\makeatother

% Define easier syntax for defining highlighted lines
% https://tex.stackexchange.com/questions/58540/highlight-lines-in-listings
\ExplSyntaxOn
\NewDocumentCommand \lstcolorlines { O{SNSorange!30} m }
{
	\clist_if_in:nVT { #2 } { \the\value{lstnumber} }{ \color{#1} }
}
\ExplSyntaxOff

% Patch internal commands of lstlinebgrd to fix background color
% https://tex.stackexchange.com/questions/398633/linebackgroundcolor-overwriting-backgroundcolor-in-lstinputlisting
\makeatletter
%alternative: patch \lst@bkgcolor 
\xpatchcmd\lst@linebgrd{\color{-.}}{\lst@bkgcolor}{}{\fail}
\makeatother	% Inclue lstlinebgrd (with some fixes) for highlighting code lines

\lstdefinestyle{thesisstyle}{
	backgroundcolor=\color{maincolor!5!white},
	commentstyle=\bfseries\itshape\color{maincolor},
	keywordstyle=\bfseries\color{maincolor},
	numberstyle=\tiny\color{maincolor},
	stringstyle=\bfseries\color{thirdcolor},
	basicstyle=\ttfamily\footnotesize,
  xleftmargin=3.2ex,
	breakatwhitespace=false,
	breaklines=true,
	captionpos=t,
	keepspaces=true,
	numbers=left,
	numbersep=7pt,
	showspaces=false,
	showstringspaces=false,
	showtabs=false,
	tabsize=4,
  escapeinside={(@}{@)},
  rulecolor=\color{maincolor},
}
\lstset{style=thesisstyle}

% ===================================================================
% CITATION

% CHANGE IN THE KAOBILBIO STYLE!!!
% DO NOT CLEARFIELD FOR archivePrefix, arxivId, and eprint!!!
% NEED TO DO THIS IF UPDATE THE KAOBOOK TEMPLATE!!!

\usepackage[
	bibstyle=numeric,			% Use numeric citations in bibliography, e.g. [5]
	citestyle=numeric-comp,		% Use compact numeric citations, e.g. [1-5,7]
	sorting=nyt,				% Sort Bibliography by name then year then title (alternative: none=citation order)
	maxnames=99,				% Show a maximum of 99 author names in Bibliography
	mincitenames=1,				% Show at least two author names when citing ...
	maxcitenames=2,				% ... But never more than two
	sortcites=true,				% Automatically sort citations numerically, e.g. [5,1,3] -> [1,3,5]
	date=year,					% Printed dates only show year
	abbreviate=false,			% Don't abbreviate string such as editor -> ed. or Tech. rep.
	% Hide some information generally:
	isbn=false,
	doi=false,
	related=false,
]{preamble/kaobiblio}

\usepackage{xpatch}				% Patch to customize the look of the Bibliography

\addbibresource{bibliography/bibliography.bib}	% Bibliography file

% Customize what appears in the margin citation (removed the ":")
\renewcommand{\formatmargincitation}[1]{%
	\parencite{#1} \citeauthor*{#1} (\citeyear{#1}), \citetitle{#1}%
}

% A custom command for only creating a marginnote with the citation
\newcommand{\onlysidecite}[2][]{\marginnote[#1]{%
	\parencite{#2} \citeauthor*{#2} (\citeyear{#2}), \citetitle{#2}%
}}

\renewbibmacro{in:}{}			% Remove "in:" from Bibliography

% Put quotes around the title of misc entries (e.g. arXiv papers) similar to "regular" paper
\DeclareFieldFormat[misc,book,techreport]{citetitle}{\mkbibquote{#1\isdot}}
\DeclareFieldFormat[misc,book,techreport]{title}{\mkbibquote{#1\isdot}}

% Publisher in Book in italics (similar to paper)
\DeclareListFormat{publisher}{%
	\usebibmacro{list:delim}{#1}%
	\mkbibemph{#1\isdot}
	\usebibmacro{list:andothers}}

% Currently this is missing a period to seperate it from the rest!
%% Remove paranthesis around the date of article
%\renewbibmacro*{issue+date}{%
%	\printfield{issue}%
%	\setunit*{\addspace}%
%	\usebibmacro{date}%
%	\newunit}


% ===================================================================
% TABLES

\usepackage{tabularx}			% Tables with flexible column width
\usepackage{multirow}			% Allow cells spread over multiple rows
\usepackage{colortbl}			% define BG colors of cells via \cellcolor
\usepackage{makecell}			% Multi-lined tabular cells (used in DeepOBS results table)

% Customization of makecell package
\renewcommand{\cellalign}{tl}
\renewcommand\theadalign{bc}
\renewcommand\theadfont{\bfseries}
\renewcommand\theadgape{\Gape[4pt]}
\renewcommand\cellgape{\Gape[4pt]}

% New columntypes for centering columns
\newcolumntype{C}{>{\centering\arraybackslash}p{3cm}}
\newcolumntype{Y}{>{\centering\arraybackslash}X}

\newcolumntype{R}{>{\raggedleft\arraybackslash}X}


% ===================================================================
% TODONOTES

\usepackage{blindtext}
\usepackage{lipsum}
\usepackage{xargs}				% Use more than one optional parameter in a new commands

\PassOptionsToPackage{colorinlistoftodos,prependcaption}{todonotes}
\newcommandx{\unsure}[2][1=]{\todo[linecolor=secondcolor,backgroundcolor=secondcolor!25,bordercolor=secondcolor,size=\footnotesize,#1]{\textbf{Unsure:}\xspace#2}}
\newcommandx{\change}[2][1=]{\todo[linecolor=maincolor,backgroundcolor=maincolor!25,bordercolor=maincolor,size=\footnotesize,#1]{\textbf{Change:}\xspace#2}}
\newcommandx{\info}[2][1=]{\todo[linecolor=darkgraycolor,backgroundcolor=darkgraycolor!25,bordercolor=darkgraycolor,size=\footnotesize,#1]{\textbf{Info:}\xspace#2}}
\newcommandx{\improvement}[2][1=]{\todo[linecolor=thirdcolor,backgroundcolor=thirdcolor!25,bordercolor=thirdcolor,size=\footnotesize,#1]{\textbf{Improve:}\xspace#2}}

% ===================================================================
% CUSTOMIZATIONS

% Only show sections in margin TOC (don't show subsections, etc.)
\setcounter{margintocdepth}{\sectiontocdepth}

% Move section numbers in margin
%\newcommand*{\numberinmargin}[1]{%
%	\makebox[0pt][r]{#1\autodot\hskip\marginparsep}}
%
%\renewcommand*{\sectionformat}{\numberinmargin{\textcolor{lightgraycolor}{\thesection}}}
%\renewcommand*{\subsectionformat}{\numberinmargin{\textcolor{lightgraycolor}{\thesubsection}}}

% OR
% Color section number in gray
\renewcommand*{\sectionformat}{\textcolor{maincolor}{\thesection}\enskip}
\renewcommand*{\subsectionformat}{\textcolor{maincolor}{\thesubsection}\enskip}

% Change style of the headers (chapter and section titles)
\renewcommand*{\chaptermarkformat}{\textbf{\chapapp~\thechapter}\enskip\color{maincolor}}
\renewcommand*{\sectionmarkformat}{\textbf{\thesection}\enskip\color{maincolor}}

% Call TOC "Table of Contents" instead of "Contents"
\addto\captionsenglish{% Replace "english" with the language you use
	\renewcommand{\contentsname}%
	{Table of Contents}%
}

% TOC Style of Parts (add color)
\newcommand\tocpartstyle[1]
{\scshape\large\bfseries\textcolor{maincolor}{#1}}
\DeclareTOCStyleEntries[pagenumberwidth=2.5em, entryformat=\tocpartstyle]{tocline}{part}%

% Rename Listing to Algorithm
\renewcommand{\lstlistingname}{Procedure}% Listing -> Procedure
\crefname{listing}{procedure}{Procedure}


% ===================================================================
% OTHER (ordered here for special reasons)

\usepackage{csquotes}				% English quotes
\usepackage{scrhack}
\usepackage{xurl}					% Allow line-breaks in URLs (loaded after biblatex)

% ===================================================================
% INPUT

%\hyphenation{au-ton-omous}

%%% Local Variables:
%%% mode: latex
%%% TeX-master: "../thesis"
%%% End:
	% Custom hyphenations
% ===================================================================
% MATH
% ===================================================================
\newcommand{\verylongrightarrow}{\xrightarrow{\hspace*{1.5cm}}}
\newcommand{\explainmath}[1]{\ensuremath{&&{\footnotesize\text{#1}}}} % Only works in align environments

% punctuation of equations, see for example the first answer in
% https://www.reddit.com/r/LaTeX/comments/5xnzg7/correct_grammar_for_putting_an_equation/
\newcommand{\equationPunctuation}[1]{\,{#1}}

% ===================================================================
% REFERENCES
% ===================================================================
\makeatletter
% TODO Allow arbitrary number of arguments in \subfigref
% https://davidyat.es/2016/07/27/writing-a-latex-macro-that-takes-a-variable-number-of-arguments/
% \newcommand{\subfigref}[1]{\textbf{(\subref{#1}\subfigrefchecknextarg)}}
% \newcommand{\subfigrefchecknextarg}{\@ifnextchar\bgroup{\subfigrefgobblenextarg}{}}
% \newcommand{\subfigrefgobblenextarg}[1]{, \subref{#1}\@ifnextchar\bgroup{\subfigrefgobblenextarg}{}}

\newcommand{\subfigref}[1]{\textbf{(\subref{#1})}}

% ===================================================================
% COMMON ABBREVIATONS

\newcommand{\PhD}{Ph.D.\@\xspace}
\newcommand*{\citeeg}{\textit{e.\nobreak\hairsp{}g.},}		% Special \eg command used in citations (doesn't result in weird space)
%\newcommand*{\ie}{i.e.\@\xspace}
\newcommand*{\cf}{cf.\@\xspace}
%\newcommand*{\etal}{et. al.\@\xspace}
%\makeatletter\newcommand*{\etc}{%
%	\@ifnextchar{.}%
%	{etc}%
%	{etc.\@\xspace}%
%}
\makeatother

\newcommand{\hl}[1]{\textbf{#1}}			% Highlight something. Alternative to \emph

\usepackage[super]{nth}						% For things like 2nd, 3rd, etc.

% ===================================================================
% SHORTHAND

\newcommand{\ai}{artificial intelligence\xspace}
\newcommand{\AI}{Artificial intelligence\xspace}
\newcommand{\ml}{machine learning\xspace}
\newcommand{\ML}{Machine learning\xspace}
\newcommand{\MLabbr}{ML\xspace}
\newcommand{\dl}{deep learning\xspace}
\newcommand{\DL}{Deep learning\xspace}
\newcommand{\DLabbr}{DL\xspace}
\newcommand{\dataset}{data set\xspace}
\newcommand{\datasets}{data sets\xspace}
\newcommand{\runtime}{runtime\xspace}
\newcommand{\mvp}{matrix-vector product\xspace}
\newcommand{\mvps}{matrix-vector products\xspace}
\newcommand{\earlystopping}{early stopping\xspace}
\newcommand{\Earlystopping}{Early stopping\xspace}


% ===================================================================
% COLORED ITEMS

% Coloured dots
\DeclareRobustCommand{\colordot}[1]{%
	\begin{tikzpicture}[baseline=(a.south)]
			\node[circle, scale=0.75,color=white, fill=#1] (a) {};
	\end{tikzpicture}%
}

% Coloured square
\DeclareRobustCommand{\colorsquare}[1]{%
	\begin{tikzpicture}[baseline=(a.south)]
			\node[rectangle, scale=0.9,color=white, fill=#1] (a) {};
	\end{tikzpicture}%
}

% Coloured line
\DeclareRobustCommand{\colorline}[1]{%
	\begin{tikzpicture}
		\raisebox{1.5pt}{
			\draw[#1,solid,line width=1.5pt] (0,0) -- (1em,0);
		}
	\end{tikzpicture}%
}

% Coloured arrow
\DeclareRobustCommand{\colorarrow}[1]{%
	\begin{tikzpicture}
		\raisebox{2.5pt}{
			\draw[#1, thick, -stealth] (0,0) -- (1em,0);
		}
	\end{tikzpicture}%
}

% Coloured shade
\newcommand{\colorshade}[1]{\textcolor{#1}{\ding{122}}}


% ===================================================================
% NUMBERS

\newcommand{\nopts}{$15$\xspace}					% Number of benchmarked optimizer
\newcommand{\noptstext}{fifteen\xspace}				% -"- in words
\newcommand{\nruns}{\mbox{$53$,$760$}\xspace}		% Number of total benchmark runs
\newcommand{\nrunsapprox}{\mbox{$50$,$000$}\xspace}	% Rounded number of total benchmark runs
\newcommand{\nrunsapproxbold}{\mbox{$\mathbf{50}$,$\mathbf{000}$}\xspace}	% -"- + bold (for titles)
\newcommand{\nconfigs}{\mbox{$1$,$920$}\xspace}		% Number of benchmarked configurations
\newcommand{\ntests}{\mbox{$128$}\xspace}			% Number of benchmarked test per optimizer
\newcommand{\nreruns}{\mbox{ten}\xspace}			% Number of re-runs with different seeds.


% ===================================================================
% PACKAGES

% Personal Packages
\newcommand{\cockpit}{Cockpit\xspace}
\newcommand{\deepobs}{DeepOBS\xspace}
\newcommand{\DeepOBS}{\deepobs}
% title versions
\newcommand{\cockpittitle}{Cockpit\xspace}
\newcommand{\deepobstitle}{DeepOBS\xspace}

% Group Packages
\newcommand{\backpack}{BackPACK\xspace}
\newcommand{\BackPACK}{\backpack}
\newcommand{\backpacktitle}{BackPACK\xspace}
\newcommand{\vivit}{ViViT\xspace}
\newcommand{\bfvivit}{\textbf{ViViT}\xspace}
\newcommand{\vivittitle}{ViViT\xspace} % Special command for the title

% General Tools & Packages
\newcommand{\python}{Python\xspace}
\newcommand{\matplotlib}{matplotlib\xspace}
\newcommand{\numpy}{NumPy\xspace}
\newcommand{\pytorch}{PyTorch\xspace}
\newcommand{\PyTorch}{\pytorch}
\newcommand{\pytorchtitle}{PyTorch\xspace}
\newcommand{\pyhessian}{PyHessian\xspace}
\newcommand{\taskset}{TaskSet\xspace}
\newcommand{\tensorboard}{TensorBoard\xspace}
\newcommand{\tensorflowdataset}{TensorFlow Datasets\xspace}
\newcommand{\tensorflow}{TensorFlow\xspace}
\newcommand{\TensorFlow}{\tensorflow}
\newcommand{\jax}{JAX\xspace}
\newcommand{\torchvision}{torchvision\xspace}
\newcommand{\wandb}{Weights \& Biases\xspace}
\newcommand{\pgfplots}{pgfplots\xspace}
\newcommand{\caffe}{Caffe\xspace}
\newcommand{\chainer}{Chainer\xspace}
\newcommand{\Chainer}{\chainer}
\newcommand{\MXNet}{\textsc{MXNet}\xspace}
\newcommand{\theano}{Theano\xspace}
\newcommand{\lasagne}{Lasagne\xspace}
\newcommand{\cuda}{CUDA\xspace}


% ===================================================================
% URLs

\newcommand{\deepobsurl}{\url{https://github.com/fsschneider/deepobs}\xspace}
\newcommand{\deepobsdocs}{\url{https://deepobs.readthedocs.io/}\xspace}
\newcommand{\benchmarkurl}{\url{https://github.com/SirRob1997/Crowded-Valley---Results}\xspace}
\newcommand{\cockpiturl}{\url{https://github.com/f-dangel/cockpit}\xspace}
\newcommand{\cockpitexpurl}{\url{https://github.com/fsschneider/cockpit-experiments}\xspace}
\newcommand{\cockpitdocurl}{\url{https://cockpit.readthedocs.io/en/latest/}\xspace}
\newcommand{\viviturl}{\url{https://github.com/jbzrE7bp/vivit}\xspace}


% ===================================================================
% OPTIMIZERS

\newcommand{\adabelief}{AdaBelief\xspace}
\newcommand{\adabound}{AdaBound\xspace}
\newcommand{\adadelta}{Adadelta\xspace}
\newcommand{\adagrad}{AdaGrad\xspace}
\newcommand{\adam}{Adam\xspace}
\newcommand{\amsbound}{AMSBound\xspace}
\newcommand{\amsgrad}{AMSGrad\xspace}
\newcommand{\lamom}{LA(Mom.)\xspace}
\newcommand{\laradam}{LA(RAdam)\xspace}
\newcommand{\ranger}{Ranger\xspace}
\newcommand{\lookaheadopt}{Lookahead\xspace}
\newcommand{\momentum}{Momentum\xspace}
\newcommand{\nadam}{Nadam\xspace}
\newcommand{\nag}{NAG\xspace}
\newcommand{\nagfull}{Nesterov Accelerated Gradient\xspace}
\newcommand{\otheropt}{Other\xspace}
\newcommand{\radam}{RAdam\xspace}
\newcommand{\rmsprop}{RMSProp\xspace}
\newcommand{\sgd}{SGD\xspace}
\newcommand{\SGD}{\sgd}
\newcommand{\sgdfull}{Stochastic Gradient Descent\xspace}

\newcommand{\gd}{GD\xspace}
\newcommand{\gdfull}{Gradient Descent\xspace}
\newcommand{\newtonsmethod}{Newton's method\xspace}

\newcommand{\lars}{LARS\xspace}
\newcommand{\lamb}{LAMB\xspace}
\newcommand{\adamw}{AdamW\xspace}

\newcommand{\cg}{CG\xspace}
\newcommand{\cgfull}{conjugate gradient\xspace}

\newcommand{\ggn}{GGN\xspace}
\newcommand{\GGN}{\ggn}
\newcommand{\ggnfull}{generalized Gauss-Newton\xspace}

\newcommand{\lbfgs}{L-BFGS\xspace}
\newcommand{\kbfgs}{K-BFGS\xspace}
\newcommand{\kfac}{K-FAC\xspace}
\newcommand{\kfra}{KFRA\xspace}
\newcommand{\shampoo}{Shampoo\xspace}
\newcommand{\adahessian}{ADAHESSIAN\xspace}

\newcommand{\sam}{SAM\xspace}
\newcommand{\samfull}{Sharpness-Aware Minimization\xspace}
\newcommand{\entropysgd}{Entropy-SGD\xspace}


% ===================================================================
% DATA SETS

\newcommand{\fmnist}{Fashion-MNIST\xspace}
\newcommand{\fmnistshort}{F-MNIST\xspace}
\newcommand{\imagenet}{ImageNet\xspace}
\newcommand{\mnist}{MNIST\xspace}
\newcommand{\svhn}{SVHN\xspace}
\newcommand{\tolstoi}{Tolstoi\xspace}
\newcommand{\warandpeace}{War and Peace\xspace}
\newcommand{\squad}{SQuAD\xspace}
\newcommand{\wmt}{WMT 2016\xspace}
\newcommand{\afhq}{AFHQ\xspace}
\newcommand{\CIFAR}[1]{CIFAR-#1\xspace}
\newcommand{\CIFARTEN}{\CIFAR{10}}
\newcommand{\cifarten}{\CIFARTEN}
\newcommand{\CIFARHUN}{\CIFAR{100}}
\newcommand{\cifarhun}{\CIFAR{100}}
\newcommand{\cifartenhun}{\CIFAR{10/100}}
\newcommand{\MNIST}{\mnist}
\newcommand{\FMNIST}{\fmnist}


% ===================================================================
% MODELS

\newcommand{\MNISTNET}{LogReg\xspace}
\newcommand{\allcnnc}{All-CNN-C\xspace}
\newcommand{\ALLCNNC}{\allcnnc}
\newcommand{\lstm}{LSTM\xspace}
\newcommand{\LSTM}{\lstm}
\newcommand{\gru}{GRU\xspace}
\newcommand{\mlp}{MLP\xspace}
\newcommand{\resnet}{ResNet\xspace}
\newcommand{\resnetthirtytwo}{ResNet-32\xspace}
\newcommand{\resnetfifty}{ResNet-50\xspace}
\newcommand{\resnetfiftysix}{ResNet-56\xspace}
\newcommand{\resnethundred}{ResNet-101\xspace}
\newcommand{\resnethundredfifty}{ResNet-152\xspace}
\newcommand{\wrn}{Wide ResNet 16-4\xspace}
\newcommand{\threecthreed}{3c3d\xspace}
\newcommand{\CIFARTENNET}{\threecthreed}
\newcommand{\twoctwod}{2c2d\xspace}
\newcommand{\FMNISTNET}{\twoctwod}
\newcommand{\vggsixteen}{VGG16\xspace}
\newcommand{\vgg}{\vggsixteen}
\newcommand{\vggnineteen}{VGG19\xspace}
\newcommand{\rnn}{RNN\xspace}
\newcommand{\rnns}{RNNs\xspace}
\newcommand{\rnnfull}{recurrent neural network\xspace}
\newcommand{\rnnsfull}{recurrent neural networks\xspace}
\newcommand{\Rnnsfull}{Recurrent neural networks\xspace}

\newcommand{\cnn}{CNN\xspace}
\newcommand{\cnns}{CNNs\xspace}
\newcommand{\cnnsfull}{convolutional neural networks\xspace}
\newcommand{\Cnnsfull}{Convolutional neural networks\xspace}

\newcommand{\gptthree}{GPT-3\xspace}
\newcommand{\bert}{BERT\xspace}
\newcommand{\mobilenet}{MobileNet\xspace}

\newcommand{\gan}{GAN\xspace}
\newcommand{\gans}{GANs\xspace}
\newcommand{\gansfull}{generative adversarial networks\xspace}

\newcommand{\gnn}{GNN\xspace}
\newcommand{\gnns}{GNNs\xspace}
\newcommand{\gnnsfull}{graph neural networks\xspace}

\newcommand{\transformer}{transformer\xspace}
\newcommand{\transformers}{transformers\xspace}

\newcommand{\vae}{VAE\xspace}
\newcommand{\vaes}{VAEs\xspace}
\newcommand{\vaesfull}{variational autoencoder\xspace}
\newcommand{\Vaesfull}{Variational autoencoder\xspace}


% ===================================================================
% BACKPACK QUANTITIES

\newcommand{\MC}{MC\xspace}
\newcommand{\mc}{\MC}
\newcommand{\DiagGGN}{DiagGGN\xspace}
\newcommand{\DiagGGNMC}{DiagGGN-MC\xspace}
\newcommand{\KFAC}{KFAC\xspace}
\newcommand{\KFLR}{KFLR\xspace}
\newcommand{\KFRA}{KFRA\xspace}

% ===================================================================
% QUANTITIES

\newcommand{\qalpha}{\texttt{\mbox{Alpha}}\xspace}

\newcommand{\qcabs}{\texttt{\mbox{CABS}}\xspace}
\newcommand{\cabs}{\textsc{\mbox{CABS}}\xspace}

\newcommand{\qearlystopping}{\texttt{\mbox{EarlyStopping}}\xspace}

\newcommand{\qgradhistoned}{\texttt{\mbox{GradHist1d}}\xspace}
\newcommand{\qgradhisttwod}{\texttt{\mbox{GradHist2d}}\xspace}

\newcommand{\qnormtest}{\texttt{\mbox{NormTest}}\xspace}
\newcommand{\qinnertest}{\texttt{\mbox{InnerTest}}\xspace}
\newcommand{\qorthotest}{\texttt{\mbox{OrthoTest}}\xspace}

\newcommand{\qhessmaxev}{\texttt{\mbox{HessMaxEV}}\xspace}
\newcommand{\qhesstrace}{\texttt{\mbox{HessTrace}}\xspace}

\newcommand{\qtictrace}{\texttt{\mbox{TICTrace}}\xspace}
\newcommand{\qticdiag}{\texttt{\mbox{TICDiag}}\xspace}
\newcommand{\tic}{\textsc{\mbox{TIC}}\xspace}

\newcommand{\qmeangsnr}{\texttt{\mbox{MeanGSNR}}\xspace}
\newcommand{\gsnr}{\textsc{\mbox{GSNR}}\xspace}

\newcommand{\qloss}{\texttt{\mbox{Loss}}\xspace}
\newcommand{\qparams}{\texttt{\mbox{Parameters}}\xspace}
\newcommand{\qdistance}{\texttt{\mbox{Distance}}\xspace}
\newcommand{\qupdatesize}{\texttt{\mbox{UpdateSize}}\xspace}
\newcommand{\qgradnorm}{\texttt{\mbox{GradNorm}}\xspace}
\newcommand{\qtime}{\texttt{\mbox{Time}}\xspace}


% ===================================================================
% ACTIVIATONS

\newcommand{\relu}{ReLU\xspace}

% ===================================================================
% OTHER

\newcommand{\refGPU}{\textsc{\mbox{Tesla K80}} GPU\xspace}
\newcommand{\github}{\textsc{\mbox{GitHub}}\xspace}
\newcommand{\arxiv}{\mbox{\normalfont\textsc{arXiv}}\xspace}
\newcommand{\arxivtitle}{\textsc{\mbox{arXiv}}\xspace}
\newcommand{\coursera}{\mbox{\normalfont\textsc{Coursera}}\xspace}

\newcommand{\fid}{\textsc{\mbox{FID}}\xspace}
\newcommand{\fidfull}{\textsc{\mbox{Fréchet inception distance}}\xspace}

\newcommand{\alphafold}{\textsc{\mbox{AlphaFold}}\xspace}
\newcommand{\lasso}{\textsc{\mbox{Lasso}}\xspace}
\newcommand{\tikhonov}{\textsc{\mbox{Tikhonov}}\xspace}

\newcommand{\klfull}{\textsc{\mbox{Kullback}-\mbox{Leibler}}\xspace}
\newcommand{\kl}{\textsc{\mbox{KL}}\xspace}

\newcommand{\smac}{\textsc{\mbox{SMAC}}\xspace}
\newcommand{\bohb}{\textsc{\mbox{BOHB}}\xspace}
\newcommand{\tpe}{\textsc{\mbox{TPE}}\xspace}
\newcommand{\spearmint}{\textsc{\mbox{Spearmint}}\xspace}

\newcommand{\mlcommons}{\textsc{\mbox{MLCommons}}\xspace}

\newcommand{\nnsvg}{\textsc{\mbox{NN-SVG}}\xspace}

%%% Local Variables:
%%% mode: latex
%%% TeX-master: "../thesis"
%%% End:
	% General Abbreviations, code envs, etc.
\newcommand{\tensor}[1]{\ensuremath{\mathsf{#1}}}
\renewcommand{\vec}{\operatorname{vec}}
\newcommand{\D}{\mathrm{D}}
\newcommand{\Gmat}{\mathrm{G}}
\newcommand{\F}{\mathrm{F}}
\newcommand{\He}{\mathrm{H}}
\newcommand{\HeCal}{\mathcal{H}}
\newcommand{\diff}{\mathrm{d}}
\newcommand{\average}[1]{\overline{ #1 }}

\newcommand{\Loss}{\mathcal{L}}
\newcommand{\jac}{\mathrm{J}}
\newcommand{\lin}[1]{\left\langle #1 \right\rangle}

\newcommand{\nablasquaredsub}[1]{
  \nabla^{\kern -.0em \raise.1em\hbox{\tiny$2$}} {\kern -.6em \raise-.1em\hbox{\tiny$#1$}}{}
}
\newcommand{\nablasub}[1]{
  \nabla {\kern -.2em \raise-.1em\hbox{\tiny$#1$}}{}
}

% Mark sections of captions for referring to divisions of figures
\newcommand{\figleft}{{\em (Left)}}
\newcommand{\figcenter}{{\em (Center)}}
\newcommand{\figright}{{\em (Right)}}
\newcommand{\figtop}{{\em (Top)}}
\newcommand{\figbottom}{{\em (Bottom)}}
\newcommand{\captiona}{{\bfseries \color{maincolor} (a)}\xspace}
\newcommand{\captionb}{{\bfseries \color{maincolor}  (b)}\xspace}
\newcommand{\captionc}{{\bfseries \color{maincolor}  (c)}\xspace}
\newcommand{\captiond}{{\bfseries \color{maincolor}  (d)}\xspace}


% Added by me (@Adrian) to be consinstent when I do listings
\newcommand{\itemi}{{\bfseries \color{maincolor} i)}\xspace}
\newcommand{\itemii}{{\bfseries \color{maincolor} ii)}\xspace}
\newcommand{\itemiii}{{\bfseries \color{maincolor} iii)}\xspace}
\newcommand{\itemiv}{{\bfseries \color{maincolor} iv)}\xspace}
\newcommand{\itemv}{{\bfseries \color{maincolor} v)}\xspace}

% Highlight a newly defined term
\newcommand{\newterm}[1]{{\bf #1}}

% Figure reference, lower-case.
\def\figref#1{figure~\ref{#1}}
% Figure reference, capital. For start of sentence
\def\Figref#1{Figure~\ref{#1}}
\def\twofigref#1#2{figures \ref{#1} and \ref{#2}}
\def\quadfigref#1#2#3#4{figures \ref{#1}, \ref{#2}, \ref{#3} and \ref{#4}}
% Section reference, lower-case.
\def\secref#1{section~\ref{#1}}
% Section reference, capital.
\def\Secref#1{Section~\ref{#1}}
% Reference to two sections.
\def\twosecrefs#1#2{sections \ref{#1} and \ref{#2}}
% Reference to three sections.
\def\secrefs#1#2#3{sections \ref{#1}, \ref{#2} and \ref{#3}}
% Reference to an equation, lower-case.
\def\eqref#1{equation~\ref{#1}}
% Reference to an equation, upper case
\def\Eqref#1{Equation~\ref{#1}}
% A raw reference to an equation---avoid using if possible
\def\plaineqref#1{\ref{#1}}
% Reference to a chapter, lower-case.
\def\chapref#1{chapter~\ref{#1}}
% Reference to an equation, upper case.
\def\Chapref#1{Chapter~\ref{#1}}
% Reference to a range of chapters
\def\rangechapref#1#2{chapters\ref{#1}--\ref{#2}}
% Reference to an algorithm, lower-case.
\def\algref#1{algorithm~\ref{#1}}
% Reference to an algorithm, upper case.
\def\Algref#1{Algorithm~\ref{#1}}
\def\twoalgref#1#2{algorithms \ref{#1} and \ref{#2}}
\def\Twoalgref#1#2{Algorithms \ref{#1} and \ref{#2}}
% Reference to a part, lower case
\def\partref#1{part~\ref{#1}}
% Reference to a part, upper case
\def\Partref#1{Part~\ref{#1}}
\def\twopartref#1#2{parts \ref{#1} and \ref{#2}}

\def\ceil#1{\lceil #1 \rceil}
\def\floor#1{\lfloor #1 \rfloor}
\def\1{\bm{1}}
\newcommand{\train}{\mathcal{D}}
\newcommand{\valid}{\mathcal{D_{\mathrm{valid}}}}
\newcommand{\test}{\mathcal{D_{\mathrm{test}}}}

\def\eps{{\epsilon}}


% Random variables
\def\reta{{\textnormal{$\eta$}}}
\def\ra{{\textnormal{a}}}
\def\rb{{\textnormal{b}}}
\def\rc{{\textnormal{c}}}
\def\rd{{\textnormal{d}}}
\def\re{{\textnormal{e}}}
\def\rf{{\textnormal{f}}}
\def\rg{{\textnormal{g}}}
\def\rh{{\textnormal{h}}}
\def\ri{{\textnormal{i}}}
\def\rj{{\textnormal{j}}}
\def\rk{{\textnormal{k}}}
\def\rl{{\textnormal{l}}}
% rm is already a command, just don't name any random variables m
\def\rn{{\textnormal{n}}}
\def\ro{{\textnormal{o}}}
\def\rp{{\textnormal{p}}}
\def\rq{{\textnormal{q}}}
\def\rr{{\textnormal{r}}}
\def\rs{{\textnormal{s}}}
\def\rt{{\textnormal{t}}}
\def\ru{{\textnormal{u}}}
\def\rv{{\textnormal{v}}}
\def\rw{{\textnormal{w}}}
\def\rx{{\textnormal{x}}}
\def\ry{{\textnormal{y}}}
\def\rz{{\textnormal{z}}}

% Random vectors
\def\rvepsilon{{\mathbf{\epsilon}}}
\def\rvtheta{{\mathbf{\theta}}}
\def\rva{{\mathbf{a}}}
\def\rvb{{\mathbf{b}}}
\def\rvc{{\mathbf{c}}}
\def\rvd{{\mathbf{d}}}
\def\rve{{\mathbf{e}}}
\def\rvf{{\mathbf{f}}}
\def\rvg{{\mathbf{g}}}
\def\rvh{{\mathbf{h}}}
\def\rvu{{\mathbf{i}}}
\def\rvj{{\mathbf{j}}}
\def\rvk{{\mathbf{k}}}
\def\rvl{{\mathbf{l}}}
\def\rvm{{\mathbf{m}}}
\def\rvn{{\mathbf{n}}}
\def\rvo{{\mathbf{o}}}
\def\rvp{{\mathbf{p}}}
\def\rvq{{\mathbf{q}}}
\def\rvr{{\mathbf{r}}}
\def\rvs{{\mathbf{s}}}
\def\rvt{{\mathbf{t}}}
\def\rvu{{\mathbf{u}}}
\def\rvv{{\mathbf{v}}}
\def\rvw{{\mathbf{w}}}
\def\rvx{{\mathbf{x}}}
\def\rvy{{\mathbf{y}}}
\def\rvz{{\mathbf{z}}}

% Elements of random vectors
\def\erva{{\textnormal{a}}}
\def\ervb{{\textnormal{b}}}
\def\ervc{{\textnormal{c}}}
\def\ervd{{\textnormal{d}}}
\def\erve{{\textnormal{e}}}
\def\ervf{{\textnormal{f}}}
\def\ervg{{\textnormal{g}}}
\def\ervh{{\textnormal{h}}}
\def\ervi{{\textnormal{i}}}
\def\ervj{{\textnormal{j}}}
\def\ervk{{\textnormal{k}}}
\def\ervl{{\textnormal{l}}}
\def\ervm{{\textnormal{m}}}
\def\ervn{{\textnormal{n}}}
\def\ervo{{\textnormal{o}}}
\def\ervp{{\textnormal{p}}}
\def\ervq{{\textnormal{q}}}
\def\ervr{{\textnormal{r}}}
\def\ervs{{\textnormal{s}}}
\def\ervt{{\textnormal{t}}}
\def\ervu{{\textnormal{u}}}
\def\ervv{{\textnormal{v}}}
\def\ervw{{\textnormal{w}}}
\def\ervx{{\textnormal{x}}}
\def\ervy{{\textnormal{y}}}
\def\ervz{{\textnormal{z}}}

% Random matrices
\def\rmA{{\mathbf{A}}}
\def\rmB{{\mathbf{B}}}
\def\rmC{{\mathbf{C}}}
\def\rmD{{\mathbf{D}}}
\def\rmE{{\mathbf{E}}}
\def\rmF{{\mathbf{F}}}
\def\rmG{{\mathbf{G}}}
\def\rmH{{\mathbf{H}}}
\def\rmI{{\mathbf{I}}}
\def\rmJ{{\mathbf{J}}}
\def\rmK{{\mathbf{K}}}
\def\rmL{{\mathbf{L}}}
\def\rmM{{\mathbf{M}}}
\def\rmN{{\mathbf{N}}}
\def\rmO{{\mathbf{O}}}
\def\rmP{{\mathbf{P}}}
\def\rmQ{{\mathbf{Q}}}
\def\rmR{{\mathbf{R}}}
\def\rmS{{\mathbf{S}}}
\def\rmT{{\mathbf{T}}}
\def\rmU{{\mathbf{U}}}
\def\rmV{{\mathbf{V}}}
\def\rmW{{\mathbf{W}}}
\def\rmX{{\mathbf{X}}}
\def\rmY{{\mathbf{Y}}}
\def\rmZ{{\mathbf{Z}}}

% Elements of random matrices
\def\ermA{{\textnormal{A}}}
\def\ermB{{\textnormal{B}}}
\def\ermC{{\textnormal{C}}}
\def\ermD{{\textnormal{D}}}
\def\ermE{{\textnormal{E}}}
\def\ermF{{\textnormal{F}}}
\def\ermG{{\textnormal{G}}}
\def\ermH{{\textnormal{H}}}
\def\ermI{{\textnormal{I}}}
\def\ermJ{{\textnormal{J}}}
\def\ermK{{\textnormal{K}}}
\def\ermL{{\textnormal{L}}}
\def\ermM{{\textnormal{M}}}
\def\ermN{{\textnormal{N}}}
\def\ermO{{\textnormal{O}}}
\def\ermP{{\textnormal{P}}}
\def\ermQ{{\textnormal{Q}}}
\def\ermR{{\textnormal{R}}}
\def\ermS{{\textnormal{S}}}
\def\ermT{{\textnormal{T}}}
\def\ermU{{\textnormal{U}}}
\def\ermV{{\textnormal{V}}}
\def\ermW{{\textnormal{W}}}
\def\ermX{{\textnormal{X}}}
\def\ermY{{\textnormal{Y}}}
\def\ermZ{{\textnormal{Z}}}

% Vectors
\def\vzero{{\bm{0}}}
\def\vone{{\bm{1}}}
\def\vmu{{\bm{\mu}}}
\def\vnu{{\bm{\nu}}}
\def\vtheta{{\bm{\theta}}}
\def\vgamma{{\bm{\gamma}}}
\def\vdelta{{\bm{\delta}}}
\def\vDelta{{\bm{\Delta}}}
\def\vepsilon{{\bm{\epsilon}}}
\def\va{{\bm{a}}}
\def\vb{{\bm{b}}}
\def\vc{{\bm{c}}}
\def\vd{{\bm{d}}}
\def\ve{{\bm{e}}}
\def\vehat{\bm{\hat{e}}}
\def\vetilde{\bm{\tilde{e}}}
\def\vell{{\bm{\ell}}}
\def\vf{{\bm{f}}}
\def\vfhat{{\bm{\hat{f}}}}
\def\vg{{\bm{g}}}
\def\vgtilde{\bm{\tilde{g}}}
\def\vghat{\bm{\hat{g}}}
\def\vh{{\bm{h}}}
\def\vi{{\bm{i}}}
\def\vj{{\bm{j}}}
\def\vk{{\bm{k}}}
\def\vl{{\bm{l}}}
\def\vm{{\bm{m}}}
\def\vmhat{{\bm{\hat{m}}}}
\def\vn{{\bm{n}}}
\def\vo{{\bm{o}}}
\def\vp{{\bm{p}}}
\def\vphi{{\bm{\phi}}}
\def\vq{{\bm{q}}}
\def\vr{{\bm{r}}}
\def\vs{{\bm{s}}}
\def\vstilde{\bm{\tilde{s}}}
\def\vt{{\bm{t}}}
\def\vu{{\bm{u}}}
\def\vv{{\bm{v}}}
\def\vvhat{{\bm{\hat{v}}}}
\def\vw{{\bm{w}}}
\def\vx{{\bm{x}}}
\def\vy{{\bm{y}}}
\def\vytilde{{\bm{\tilde{y}}}}
\def\vyhat{{\bm{\hat{y}}}}
\def\vz{{\bm{z}}}

% Elements of vectors
\def\evalpha{{\alpha}}
\def\evbeta{{\beta}}
\def\evepsilon{{\epsilon}}
\def\evlambda{{\lambda}}
\def\evomega{{\omega}}
\def\evmu{{\mu}}
\def\evpsi{{\psi}}
\def\evsigma{{\sigma}}
\def\evtheta{{\theta}}
\def\eva{{a}}
\def\evb{{b}}
\def\evc{{c}}
\def\evd{{d}}
\def\eve{{e}}
\def\evf{{f}}
\def\evg{{g}}
\def\evh{{h}}
\def\evi{{i}}
\def\evj{{j}}
\def\evk{{k}}
\def\evl{{l}}
\def\evm{{m}}
\def\evn{{n}}
\def\evo{{o}}
\def\evp{{p}}
\def\evq{{q}}
\def\evr{{r}}
\def\evs{{s}}
\def\evt{{t}}
\def\evu{{u}}
\def\evv{{v}}
\def\evw{{w}}
\def\evx{{x}}
\def\evy{{y}}
\def\evyhat{{\hat{y}}}
\def\evz{{z}}

% Matrix
\def\mA{{\bm{A}}}
\def\mB{{\bm{B}}}
\def\mC{{\bm{C}}}
\def\mD{{\bm{D}}}
\def\mE{{\bm{E}}}
\def\mF{{\bm{F}}}
\def\mG{{\bm{G}}}
\def\mGtilde{\bm{\tilde{G}}}
\def\mH{{\bm{H}}}
\def\mI{{\bm{I}}}
\def\mJ{{\bm{J}}}
\def\mK{{\bm{K}}}
\def\mL{{\bm{L}}}
\def\mM{{\bm{M}}}
\def\mN{{\bm{N}}}
\def\mO{{\bm{O}}}
\def\mP{{\bm{P}}}
\def\mQ{{\bm{Q}}}
\def\mR{{\bm{R}}}
\def\mS{{\bm{S}}}
\def\mT{{\bm{T}}}
\def\mU{{\bm{U}}}
\def\mV{{\bm{V}}}
\def\mW{{\bm{W}}}
\def\mWhat{{\bm{\hat{W}}}}
\def\mX{{\bm{X}}}
\def\mY{{\bm{Y}}}
\def\mZ{{\bm{Z}}}
\def\mBeta{{\bm{\beta}}}
\def\mPhi{{\bm{\Phi}}}
\def\mPi{{\bm{\Pi}}}
\def\mLambda{{\bm{\Lambda}}}
\def\mSigma{{\bm{\Sigma}}}
\def\mStilde{\bm{\tilde{\mS}}}
\def\mGtilde{\bm{\tilde{\mG}}}
\def\mGoverline{{\bm{\overline{G}}}}

% Tensor
\DeclareMathAlphabet{\mathsfit}{\encodingdefault}{\sfdefault}{m}{sl}
\SetMathAlphabet{\mathsfit}{bold}{\encodingdefault}{\sfdefault}{bx}{n}
\newcommand{\tens}[1]{\bm{\mathsfit{#1}}}
\def\tA{{\tens{A}}}
\def\tB{{\tens{B}}}
\def\tC{{\tens{C}}}
\def\tD{{\tens{D}}}
\def\tE{{\tens{E}}}
\def\tF{{\tens{F}}}
\def\tG{{\tens{G}}}
\def\tH{{\tens{H}}}
\def\tI{{\tens{I}}}
\def\tJ{{\tens{J}}}
\def\tK{{\tens{K}}}
\def\tL{{\tens{L}}}
\def\tM{{\tens{M}}}
\def\tN{{\tens{N}}}
\def\tO{{\tens{O}}}
\def\tP{{\tens{P}}}
\def\tQ{{\tens{Q}}}
\def\tR{{\tens{R}}}
\def\tS{{\tens{S}}}
\def\tT{{\tens{T}}}
\def\tU{{\tens{U}}}
\def\tV{{\tens{V}}}
\def\tW{{\tens{W}}}
\def\tX{{\tens{X}}}
\def\tY{{\tens{Y}}}
\def\tZ{{\tens{Z}}}


% Graph
\def\gA{{\mathcal{A}}}
\def\gB{{\mathcal{B}}}
\def\gC{{\mathcal{C}}}
\def\gD{{\mathcal{D}}}
\def\gE{{\mathcal{E}}}
\def\gF{{\mathcal{F}}}
\def\gG{{\mathcal{G}}}
\def\gH{{\mathcal{H}}}
\def\gI{{\mathcal{I}}}
\def\gJ{{\mathcal{J}}}
\def\gK{{\mathcal{K}}}
\def\gL{{\mathcal{L}}}
\def\gM{{\mathcal{M}}}
\def\gN{{\mathcal{N}}}
\def\gO{{\mathcal{O}}}
\def\gP{{\mathcal{P}}}
\def\gQ{{\mathcal{Q}}}
\def\gR{{\mathcal{R}}}
\def\gS{{\mathcal{S}}}
\def\gT{{\mathcal{T}}}
\def\gU{{\mathcal{U}}}
\def\gV{{\mathcal{V}}}
\def\gW{{\mathcal{W}}}
\def\gX{{\mathcal{X}}}
\def\gY{{\mathcal{Y}}}
\def\gZ{{\mathcal{Z}}}

% Sets
\def\sA{{\mathbb{A}}}
\def\sB{{\mathbb{B}}}
\def\sC{{\mathbb{C}}}
\def\sD{{\mathbb{D}}}
% Don't use a set called E, because this would be the same as our symbol
% for expectation.
\def\sF{{\mathbb{F}}}
\def\sG{{\mathbb{G}}}
\def\sH{{\mathbb{H}}}
\def\sI{{\mathbb{I}}}
\def\sJ{{\mathbb{J}}}
\def\sK{{\mathbb{K}}}
\def\sL{{\mathbb{L}}}
\def\sM{{\mathbb{M}}}
\def\sN{{\mathbb{N}}}
\def\sO{{\mathbb{O}}}
\def\sP{{\mathbb{P}}}
\def\sQ{{\mathbb{Q}}}
\def\sR{{\mathbb{R}}}
\def\sS{{\mathbb{S}}}
\def\sT{{\mathbb{T}}}
\def\sU{{\mathbb{U}}}
\def\sV{{\mathbb{V}}}
\def\sW{{\mathbb{W}}}
\def\sX{{\mathbb{X}}}
\def\sY{{\mathbb{Y}}}
\def\sYhat{{\hat{\mathbb{Y}}}}
\def\sZ{{\mathbb{Z}}}
% Blackboard Greek characters: https://tex.stackexchange.com/a/3260
\DeclareSymbolFont{bbold}{U}{bbold}{m}{n}
\DeclareSymbolFontAlphabet{\mathbbold}{bbold}
\def\sOmega{{\mathbbold{\Omega}}}
\def\sPhi{{\mathbbold{\Phi}}}
\def\sTheta{{\mathbbold{\Theta}}}

% Entries of a matrix
\def\emLambda{{\Lambda}}
\def\emA{{A}}
\def\emB{{B}}
\def\emC{{C}}
\def\emD{{D}}
\def\emE{{E}}
\def\emF{{F}}
\def\emG{{G}}
\def\emH{{H}}
\def\emI{{I}}
\def\emJ{{J}}
\def\emK{{K}}
\def\emL{{L}}
\def\emM{{M}}
\def\emN{{N}}
\def\emO{{O}}
\def\emP{{P}}
\def\emQ{{Q}}
\def\emR{{R}}
\def\emS{{S}}
\def\emT{{T}}
\def\emU{{U}}
\def\emV{{V}}
\def\emW{{W}}
\def\emX{{X}}
\def\emY{{Y}}
\def\emZ{{Z}}
\def\emSigma{{\Sigma}}
\def\emPi{{\Pi}}

% entries of a tensor
% Same font as tensor, without \bm wrapper
\newcommand{\etens}[1]{\mathsfit{#1}}
\def\etLambda{{\etens{\Lambda}}}
\def\etA{{\etens{A}}}
\def\etB{{\etens{B}}}
\def\etC{{\etens{C}}}
\def\etD{{\etens{D}}}
\def\etE{{\etens{E}}}
\def\etF{{\etens{F}}}
\def\etG{{\etens{G}}}
\def\etH{{\etens{H}}}
\def\etI{{\etens{I}}}
\def\etJ{{\etens{J}}}
\def\etK{{\etens{K}}}
\def\etL{{\etens{L}}}
\def\etM{{\etens{M}}}
\def\etN{{\etens{N}}}
\def\etO{{\etens{O}}}
\def\etP{{\etens{P}}}
\def\etQ{{\etens{Q}}}
\def\etR{{\etens{R}}}
\def\etS{{\etens{S}}}
\def\etT{{\etens{T}}}
\def\etU{{\etens{U}}}
\def\etV{{\etens{V}}}
\def\etW{{\etens{W}}}
\def\etX{{\etens{X}}}
\def\etY{{\etens{Y}}}
\def\etZ{{\etens{Z}}}

% The true underlying data generating distribution
\newcommand{\pdata}{p_{\mathrm{data}}}
% The empirical distribution defined by the training set
\newcommand{\ptrain}{\hat{p}_{\mathrm{data}}}
\newcommand{\Ptrain}{\hat{P}_{\mathrm{data}}}
% The model distribution
\newcommand{\pmodel}{p_{\rm{model}}}
\newcommand{\Pmodel}{P_{\rm{model}}}
\newcommand{\ptildemodel}{\tilde{p}_{\rm{model}}}
% Stochastic autoencoder distributions
\newcommand{\pencode}{p_{\rm{encoder}}}
\newcommand{\pdecode}{p_{\rm{decoder}}}
\newcommand{\precons}{p_{\rm{reconstruct}}}

\ifcsname laplace\endcsname%
    \renewcommand{\laplace}{\mathrm{Laplace}}% Laplace distribution
\else%
    \newcommand{\laplace}{\mathrm{Laplace}}% Laplace distribution
\fi

%\newcommand{\E}{\mathbb{E}}
\DeclareMathOperator{\E}{\mathbb{E}}

\newcommand{\Ls}{\mathcal{L}}
\newcommand{\R}{\mathbb{R}}
\newcommand{\emp}{\tilde{p}}
\newcommand{\lr}{\alpha}
\newcommand{\reg}{\lambda}
\newcommand{\rect}{\mathrm{rectifier}}
\newcommand{\softmax}{\mathrm{softmax}}
\newcommand{\onehot}{\mathrm{onehot}}
\newcommand{\sigmoid}{\sigma}
\newcommand{\softplus}{\zeta}
%\newcommand{\KL}{D_{\mathrm{KL}}}
\DeclareMathOperator{\KL}{KL}
%\newcommand{\Var}{\mathrm{Var}}
\newcommand{\standarderror}{\mathrm{SE}}
\newcommand{\Cov}{\mathrm{Cov}}
% Wolfram Mathworld says $L^2$ is for function spaces and $\ell^2$ is for vectors
% But then they seem to use $L^2$ for vectors throughout the site, and so does
% wikipedia.
\newcommand{\normlzero}{L^0}
\newcommand{\normlone}{L^1}
\newcommand{\normltwo}{L^2}
\newcommand{\normlp}{L^p}
\newcommand{\normmax}{L^\infty}

\newcommand{\parents}{Pa} % See usage in notation.tex. Chosen to match Daphne's book.

\DeclareMathOperator*{\argmax}{arg\,max}
\DeclareMathOperator*{\argmin}{arg\,min}
\DeclareMathOperator*{\minimize}{minimize}
\DeclareMathOperator*{\maximize}{maximize}

\DeclareMathOperator{\sign}{sign}
%\DeclareMathOperator{\mean}{mean}
\DeclareMathOperator{\vmap}{vmap}
\DeclareMathOperator{\reshape}{reshape}
\DeclareMathOperator{\Tr}{Tr}
\DeclareMathOperator{\diag}{diag}
\DeclareMathOperator{\eig}{eig}
\DeclareMathOperator{\rank}{rank}
\DeclareMathOperator{\vecspan}{span}
\DeclareMathOperator{\overlap}{overlap}
\DeclareMathOperator{\Cat}{Cat}

%%%%% NEW MATH DEFINITIONS %%%%%
\newcommand{\grad}[1]{\ensuremath{\nabla_{\!{#1}}}}
\newcommand{\naturalgrad}[1]{\ensuremath{\tilde{\nabla}_{\!{#1}}}}
\newcommand{\gradsquared}[1]{\ensuremath{\nabla_{\!{#1}}^{2}}}
\newcommand{\hess}[1]{\ensuremath{\mathrm{H}_{#1}}}

\newcommand{\Dtrain}{\ensuremath{\sD_{\mathrm{train}}}}
\newcommand{\Dtest}{\ensuremath{\sD_{\mathrm{test}}}}
\newcommand{\Dvalid}{\ensuremath{\sD_{\mathrm{valid}}}}

% || for KLDivergence
\DeclarePairedDelimiterX{\KLdivx}[2]{(}{)}{%
  #1\;\delimsize\|\;#2%
}
\newcommand{\KLdiv}{\KL\KLdivx}

% | for a | b
%\newcommand{\giventhat}[2]{#1\;|\;#2}

\NewDocumentCommand{\giventhat}{O{\;|} m m}{#2#1\;#3}

\let\ab\allowbreak

%%% Local Variables:
%%% mode: latex
%%% TeX-master: "../thesis"
%%% End:
 % Math Commands from the Deep Learning book

% ===================================================================
% HOTFIXES
\renewcommand*{\figureformat}{%
  \figurename~\thefigure%
  % \autodot% DELETED
}
\renewcommand*{\tableformat}{%
  \tablename~\thetable%
  % \autodot% DELETED
}

%%% Local Variables:
%%% mode: latex
%%% TeX-master: "../thesis"
%%% End:
